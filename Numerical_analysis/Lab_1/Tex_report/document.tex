\documentclass[14pt,a4paper]{scrartcl}
\usepackage{cmap}
\usepackage[utf8]{inputenc}
\usepackage[T1,T2A]{fontenc}
\usepackage[english,russian]{babel}
\usepackage{relsize}
\usepackage{graphicx}
\usepackage{subfigure}
\usepackage{mathtools}
\usepackage{amssymb}
\usepackage{float}
\usepackage{sidecap}
\usepackage{wrapfig}
\usepackage{caption}
\usepackage[table,xcdraw]{xcolor}
\usepackage{listings}
\usepackage{amsmath,cryptocode}
\usepackage{listings}

\newcommand\scalemath[2]{\scalebox{#1}{\mbox{\ensuremath{\displaystyle #2}}}}


\begin{document}
	\begin{titlepage}
	\begin{center}
		\large
		МИНИСТЕРСТВО ОБРАЗОВАНИЯ И НАУКИ\\ РОССИЙСКОЙ ФЕДЕРАЦИИ
		
		\vspace{0.5cm}
		
		МГТУ им Н.Э.Баумана
		\vspace{0.25cm}
		
		Факультет ФН
		
		Кафедра вычислительной математики и математической физики
		\vfill
		
		
		Соколов Арсений Андреевич\\
		\vfill
		
		
		{\LARGE Лабораторная работа №1 по численным методам\\[2mm]
		}
		\bigskip
		
		3 курс, группа ФН11-53Б\\
		Вариант 9
	\end{center}
	\vfill
	
	\newlength{\ML}
	\settowidth{\ML}{«\underline{\hspace{0.7cm}}» \underline{\hspace{2cm}}}
	\hfill\begin{minipage}{0.4\textwidth}
		Преподаватель\\
		\underline{\hspace{3cm}} В.\,А.~Кутыркин\\
		«\underline{\hspace{0.7cm}}» \underline{\hspace{1.71cm}} 2019 г.
	\end{minipage}%
	\bigskip
	
	
	\vfill
	
	\begin{center}
		Москва, 2019 г.
	\end{center}
\end{titlepage}

\section*{Задание 1.1}
\textbf{Задание.} Определить число обусловленности матрицы рассматриваемой СЛАУ и найти относительную погрешность в решении приближенной СЛАУ.
Исходная СЛАУ:
\[
	\begin{cases}
	276.5 \cdot x_1      +  275 \cdot x_2      +  275 \cdot x_3   =   826.5 \\ 
	275.55 \cdot x_1  +  275.947 \cdot x_2      +  275 \cdot x_3   =   826.5 \\ 
	274.45 \cdot x_1      +  275 \cdot x_2  +  277.053 \cdot x_3   =   826.5 \\ 
	\end{cases}
\]

Приближенная СЛАУ:
\begin{equation*}
	\begin{cases}
	276.5 \cdot x_1      +  275 \cdot x_2      +  275 \cdot x_3   =   834.765 \\ 
	275.55 \cdot x_1  +  275.947 \cdot x_2      +  275 \cdot x_3   =   818.235 \\ 
	274.45 \cdot x_1      +  275 \cdot x_2  +  277.053 \cdot x_3   =   834.765 \\ 
	\end{cases}
\end{equation*}

\textbf{Решение.}\\
 По исходной СЛАУ имеем соответствующие матрицы:
\begin{equation*}
	A = 
	\begin{bmatrix}{}
	276.500 & 275.000 & 275.000 \\ 
	275.550 & 275.947 & 275.000 \\ 
	274.450 & 275.000 & 277.053 \\ 
	\end{bmatrix}
\end{equation*}

\begin{equation*}
	\prescript{>}{}{b} = 
	\begin{bmatrix}{}
	826.500 \\ 
	826.500 \\ 
	826.500 \\ 
	\end{bmatrix}
\end{equation*}

\begin{equation*}
	A^{-1} = 
	\begin{bmatrix}{}
	0.514 & -0.351 & -0.162 \\ 
	-0.540 & 0.704 & -0.162 \\ 
	0.026 & -0.351 & 0.325 \\ 
	\end{bmatrix}
\end{equation*}


\begin{equation*}
	\prescript{>}{}{b} + \prescript{>}{}{\Delta b} = 
	\begin{bmatrix}{}
	834.765 \\ 
	818.235 \\ 
	834.765 \\ 
	\end{bmatrix}
\end{equation*}


Число обусловленности:
\begin{equation*}
	cond(A) = ||A|| \cdot ||A^{-1}|| = 826.503 \cdot 1.406193 = 1162.223
\end{equation*} 

Таким образом, матрица нашей СЛАУ плохо обусловлена. \\\\
Найдём относительную погрешность в решении приближенной СЛАУ.

\begin{equation*}
	A \cdot \prescript{>}{}{x} = \prescript{>}{}{b}
\end{equation*}

\begin{equation*}
	\prescript{>}{}{x} = A^{-1} \cdot \prescript{>}{}{b} = \begin{bmatrix}{}
	0.9994325 \\ 
	1.0025986 \\ 
	0.9979720 \\ 
	\end{bmatrix}
\end{equation*}

Ошибка $\prescript{>}{}{\Delta x} = A^{-1} \cdot \prescript{>}{}{\Delta b} = \begin{bmatrix}{}
5.8145124 \\ 
-11.6221892 \\ 
5.8060158 \\ 
\end{bmatrix} $

Относительная погрешность приближенного решения:
\begin{equation*}
	\frac{||\prescript{>}{}{\Delta x}||}{||\prescript{>}{}{x}||} = \frac{11.62219}{1.002599} = 11.59207
\end{equation*}

\begin{equation*}
	||\prescript{>}{}{\Delta b}|| = 8.265 
\end{equation*}

\begin{equation*}
	||\prescript{>}{}{b}|| = 826.5
\end{equation*}

Тогда:
\begin{equation*}
	\frac{||\prescript{>}{}{\Delta x}||}{||\prescript{>}{}{x}||} = 11.59207 \leq cond(A) \cdot \frac{||\prescript{>}{}{\Delta b}||}{||\prescript{>}{}{b}||} = 1162.223 \cdot 0.01 = 11.62223
\end{equation*}

\textbf{Результаты.}\\
Число обусловленности $cond(A) = ||A|| \cdot ||A^{-1}|| = 1162.223 > 10^2$, значит, матрица СЛАУ плохо обусловлена.\\
Относительная погрешность $	\frac{||\prescript{>}{}{\Delta x}||}{||\prescript{>}{}{x}||} = 11.59207$ очень велика вследствие плохой обусловленности матрицы СЛАУ.

\section*{Задание 1.2}
\textbf{Задание.}\\
Исходные данные:
\begin{itemize}
	\item $N = 9$
	\item $\lambda + \alpha = 0.6$
	\item $F = \arctan(x)$
	\item $a = 0$
	\item $b = 1$
\end{itemize}

Согласно этой таблице, на отрезке $[a;b]$ выбрана центрально равномерная сетка с десятью узлами:\\
$s_1 = \tau_1 = a + h/2$ , $s_2 = \tau_2 = \tau_1 + h$, \dots, $s_10 = \tau_10 = \tau_9 + h$, имеющая шаг $h = \frac{b - a}{10}$

Требуется решить приближенную СЛАУ:
\begin{equation*}
	(E + \lambda A) \prescript{>}{}{x} = \prescript{>}{}{b} + \prescript{>}{}{\Delta b},
\end{equation*}
\begin{itemize}
	\item[] $ \lambda \in \mathbb{R}$,
	\item[] $ E \in GL(\mathbb{R}, 10)$ -- единичная матрица,
	\item[] $ A = (a_j^i)_{10}^{10} \in GL(\mathbb{R}, 10)$,
	\item[] $ \prescript{>}{}{b} = [b^1, \cdots b^{10}\rangle  \in \prescript{>}{}{\mathbb{R}^{10}}$.
\end{itemize}

Причём:
\begin{equation*}
	a_j^i = F(s_i \cdot \tau_j) \frac{b-a}{10}, \qquad \text{для} i,j = \overline{1,10}
\end{equation*}
\begin{equation*}
	\prescript{>}{}{b} = (E + \lambda A)\cdot \prescript{>}{}{x}
\end{equation*}
\begin{equation*}
	\prescript{>}{}{x} = [1, \cdots 1\rangle \in \prescript{>}{}{\mathbb{R}^{10}}
\end{equation*}

Согласно СЛАУ из задания 1.1, приближенная СЛАУ определяется только погрешностью $\prescript{>}{}{b} = [b^1, \cdots b^{10}\rangle = 0.01 \cdot [b^1, -b^2, \cdots b^9, -b^{10}\rangle \in \prescript{>}{}{\mathbb{R}^{10}}$ в правой части исходной СЛАУ.

Требуется найти число обусловленности матрицы рассматриваемой СЛАУ и относительную погрешность в решении приближенной СЛАУ. Кроме того, найти решение СЛАУ, которая получается из исходной делением каждого $i-$го уравнения $(i = \overline{1,10})$ на число $b^i + \Delta b^i$. После этого сравнить абсолютную погрешность в решении получившейся СЛАУ с абсолютной погрешностью в решении приближенной СЛАУ.

\textbf{Решение.}\\
Матрица $A$:
\begin{equation*}
	\scalemath{0.75}
	{
		\begin{bmatrix}{}
		0.0002500 & 0.0007500 & 0.0012499 & 0.0017498 & 0.0022496 & 0.0027493 & 0.0032489 & 0.0037482 & 0.0042474 & 0.0047464 \\ 
		0.0007500 & 0.0022496 & 0.0037482 & 0.0052452 & 0.0067398 & 0.0082314 & 0.0097193 & 0.0112029 & 0.0126816 & 0.0141547 \\ 
		0.0012499 & 0.0037482 & 0.0062419 & 0.0087278 & 0.0112029 & 0.0136643 & 0.0161092 & 0.0185348 & 0.0209385 & 0.0233180 \\ 
		0.0017498 & 0.0052452 & 0.0087278 & 0.0121893 & 0.0156217 & 0.0190174 & 0.0223693 & 0.0256708 & 0.0289162 & 0.0321000 \\ 
		0.0022496 & 0.0067398 & 0.0112029 & 0.0156217 & 0.0199798 & 0.0242624 & 0.0284562 & 0.0325496 & 0.0365330 & 0.0403986 \\ 
		0.0027493 & 0.0082314 & 0.0136643 & 0.0190174 & 0.0242624 & 0.0293749 & 0.0343341 & 0.0391236 & 0.0437311 & 0.0481485 \\ 
		0.0032489 & 0.0097193 & 0.0161092 & 0.0223693 & 0.0284562 & 0.0343341 & 0.0399751 & 0.0453598 & 0.0504761 & 0.0553188 \\ 
		0.0037482 & 0.0112029 & 0.0185348 & 0.0256708 & 0.0325496 & 0.0391236 & 0.0453598 & 0.0512389 & 0.0567538 & 0.0619066 \\ 
		0.0042474 & 0.0126816 & 0.0209385 & 0.0289162 & 0.0365330 & 0.0437311 & 0.0504761 & 0.0567538 & 0.0625668 & 0.0679297 \\ 
		0.0047464 & 0.0141547 & 0.0233180 & 0.0321000 & 0.0403986 & 0.0481485 & 0.0553188 & 0.0619066 & 0.0679297 & 0.0734195 \\ 
		\end{bmatrix}
	}
\end{equation*}


Матрица $E + \lambda A$:
\begin{equation*}
\scalemath{0.75}
{
	\begin{bmatrix}{}
	1.0001425 & 0.0004275 & 0.0007125 & 0.0009974 & 0.0012823 & 0.0015671 & 0.0018518 & 0.0021365 & 0.0024210 & 0.0027055 \\ 
	0.0004275 & 1.0012823 & 0.0021365 & 0.0029898 & 0.0038417 & 0.0046919 & 0.0055400 & 0.0063857 & 0.0072285 & 0.0080682 \\ 
	0.0007125 & 0.0021365 & 1.0035579 & 0.0049748 & 0.0063857 & 0.0077887 & 0.0091822 & 0.0105648 & 0.0119350 & 0.0132912 \\ 
	0.0009974 & 0.0029898 & 0.0049748 & 1.0069479 & 0.0089044 & 0.0108399 & 0.0127505 & 0.0146324 & 0.0164822 & 0.0182970 \\ 
	0.0012823 & 0.0038417 & 0.0063857 & 0.0089044 & 1.0113885 & 0.0138296 & 0.0162200 & 0.0185533 & 0.0208238 & 0.0230272 \\ 
	0.0015671 & 0.0046919 & 0.0077887 & 0.0108399 & 0.0138296 & 1.0167437 & 0.0195704 & 0.0223004 & 0.0249267 & 0.0274447 \\ 
	0.0018518 & 0.0055400 & 0.0091822 & 0.0127505 & 0.0162200 & 0.0195704 & 1.0227858 & 0.0258551 & 0.0287714 & 0.0315317 \\ 
	0.0021365 & 0.0063857 & 0.0105648 & 0.0146324 & 0.0185533 & 0.0223004 & 0.0258551 & 1.0292062 & 0.0323496 & 0.0352868 \\ 
	0.0024210 & 0.0072285 & 0.0119350 & 0.0164822 & 0.0208238 & 0.0249267 & 0.0287714 & 0.0323496 & 1.0356631 & 0.0387200 \\ 
	0.0027055 & 0.0080682 & 0.0132912 & 0.0182970 & 0.0230272 & 0.0274447 & 0.0315317 & 0.0352868 & 0.0387200 & 1.0418491 \\ 
	\end{bmatrix}
}
\end{equation*}

Матрица $(E + \lambda A)^{-1}$:
\begin{equation*}
\scalemath{0.71}
{
	\begin{bmatrix}{}
	0.999881 & -0.000359 & -0.000599 & -0.000840 & -0.001083 & -0.001327 & -0.001574 & -0.001823 & -0.002073 & -0.002326 \\ 
	-0.000359 & 0.998923 & -0.001797 & -0.002519 & -0.003245 & -0.003975 & -0.004709 & -0.005448 & -0.006189 & -0.006934 \\ 
	-0.000599 & -0.001797 & 0.997004 & -0.004196 & -0.005399 & -0.006603 & -0.007809 & -0.009013 & -0.010216 & -0.011415 \\ 
	-0.000840 & -0.002519 & -0.004196 & 0.994130 & -0.007539 & -0.009199 & -0.010849 & -0.012484 & -0.014102 & -0.015700 \\ 
	-0.001083 & -0.003245 & -0.005399 & -0.007539 & 0.990342 & -0.011750 & -0.013809 & -0.015830 & -0.017806 & -0.019734 \\ 
	-0.001327 & -0.003975 & -0.006603 & -0.009199 & -0.011750 & 0.985755 & -0.016673 & -0.019028 & -0.021300 & -0.023487 \\ 
	-0.001574 & -0.004709 & -0.007809 & -0.010849 & -0.013809 & -0.016673 & 0.980573 & -0.022060 & -0.024566 & -0.026942 \\ 
	-0.001823 & -0.005448 & -0.009013 & -0.012484 & -0.015830 & -0.019028 & -0.022060 & 0.975081 & -0.027598 & -0.030100 \\ 
	-0.002073 & -0.006189 & -0.010216 & -0.014102 & -0.017806 & -0.021300 & -0.024566 & -0.027598 & 0.969603 & -0.032971 \\ 
	-0.002326 & -0.006934 & -0.011415 & -0.015700 & -0.019734 & -0.023487 & -0.026942 & -0.030100 & -0.032971 & 0.964428 \\ 
	\end{bmatrix}
}
\end{equation*}


Найдём число обусловленности матрицы:\\
$||E + \lambda A|| = 1.240221$,\\  $||(E + \lambda A)^{-1} || = 1.134036$,\\  $cond(E + \lambda A) = ||E + \lambda A|| \cdot ||(E + \lambda A)^{-1} || = 1.406456 $

Таким образом, матрица СЛАУ задания 1.2 хорошо обусловлена.

Решение СЛАУ: $E+\lambda A) \cdot \prescript{>}{}{x} = \prescript{>}{}{b}$, согласно условию имеет вид:
\begin{equation*}
	\prescript{>}{}{x} = [1, \cdots 1\rangle \in \prescript{>}{}{\mathbb{R}^{10}}
\end{equation*}


Поэтому
\begin{equation*}
\prescript{>}{}{b} = 
	\begin{bmatrix}{}
	1.014244 \\ 
	1.042592 \\ 
	1.070529 \\ 
	1.097816 \\ 
	1.124256 \\ 
	1.149703 \\ 
	1.174059 \\ 
	1.197271 \\ 
	1.219321 \\ 
	1.240221 \\ 
	\end{bmatrix},
	\qquad ||\prescript{>}{}{b}|| = 1.240221
\end{equation*}


Вычислим погрешность решения СЛАУ:
\begin{equation*}
	\prescript{>}{}{b} = [b^1, \cdots b^{10}\rangle = 0.01 \cdot [b^1, -b^2, \cdots b^9, -b^{10}\rangle \in \prescript{>}{}{\mathbb{R}^{10}}
\end{equation*}

\begin{equation*}
	\prescript{>}{}{b} = 
	\begin{bmatrix}{}
	0.010142 \\ 
	-0.010426 \\ 
	0.010705 \\ 
	-0.010978 \\ 
	0.011243 \\ 
	-0.011497 \\ 
	0.011741 \\ 
	-0.011973 \\ 
	0.012193 \\ 
	-0.012402 \\ 
	\end{bmatrix},
	\qquad ||\prescript{>}{}{b}|| = 0.01240221
\end{equation*}

\begin{equation*}
	\prescript{>}{}{b} + \prescript{>}{}{\Delta b} = 
	\begin{bmatrix}{}
	1.024387 \\ 
	1.032166 \\ 
	1.081235 \\ 
	1.086838 \\ 
	1.135499 \\ 
	1.138206 \\ 
	1.185800 \\ 
	1.185298 \\ 
	1.231514 \\ 
	1.227819 \\ 
	\end{bmatrix},
	\qquad ||\prescript{>}{}{b} + \prescript{>}{}{\Delta b}|| = 1.231514
\end{equation*}
\\

Решение приближенной СЛАУ:
\begin{equation*}
	\prescript{>}{}{x} + \prescript{>}{}{\Delta x} = (E + \lambda A)^{-1} \cdot (\prescript{>}{}{b} + \prescript{>}{}{\Delta b})
\end{equation*}

\begin{equation*}
	\prescript{>}{}{x} + \prescript{>}{}{\Delta x} = 
	\begin{bmatrix}{}
	1.010158 \\ 
	0.989620 \\ 
	1.010780 \\ 
	0.989125 \\ 
	1.011372 \\ 
	0.988657 \\ 
	1.011916 \\ 
	0.988223 \\ 
	1.012407 \\ 
	0.987828 \\ 
	\end{bmatrix},
	\qquad 
	\prescript{>}{}{\Delta x} =
	\begin{bmatrix}{}
	0.010158 \\ 
	-0.010380 \\ 
	0.010780 \\ 
	-0.010875 \\ 
	0.011372 \\ 
	-0.011343 \\ 
	0.011916 \\ 
	-0.011777 \\ 
	0.012407 \\ 
	-0.012172 \\ 
	\end{bmatrix}
\end{equation*}


\begin{equation*}
	|| \prescript{>}{}{x} || = 1, \qquad || \prescript{>}{}{\Delta x} || = 0.01240714
\end{equation*}

Относительная погрешность: $\frac{|| \prescript{>}{}{\Delta x} ||}{|| \prescript{>}{}{x} ||} = 0.01240714$


Действительно:\\
 $\frac{|| \prescript{>}{}{\Delta x} ||}{|| \prescript{>}{}{x} ||} = 0.01240714 \leq cond(E + \lambda A) \frac{||\prescript{>}{}{\Delta b}||}{||\prescript{>}{}{b}||} = 1.406456 \cdot 0.01 = 0.01406456$

Так как СЛАУ хорошо обусловлена, то и погрешность небольшая.

Найдём СЛАУ, которая получается делением каждой $i-$ой строки исходной матрицы на $b^i + \Delta b^i (i=\overline{1,10})$. Получим матрицу $B$:
\begin{equation*}
\scalemath{0.75}
{
	B = 
	\begin{bmatrix}{}
	0.976333 & 0.000417 & 0.000696 & 0.000974 & 0.001252 & 0.001530 & 0.001808 & 0.002086 & 0.002363 & 0.002641 \\ 
	0.000414 & 0.970079 & 0.002070 & 0.002897 & 0.003722 & 0.004546 & 0.005367 & 0.006187 & 0.007003 & 0.007817 \\ 
	0.000659 & 0.001976 & 0.928159 & 0.004601 & 0.005906 & 0.007203 & 0.008492 & 0.009771 & 0.011038 & 0.012293 \\ 
	0.000918 & 0.002751 & 0.004577 & 0.926493 & 0.008193 & 0.009974 & 0.011732 & 0.013463 & 0.015165 & 0.016835 \\ 
	0.001129 & 0.003383 & 0.005624 & 0.007842 & 0.890700 & 0.012179 & 0.014284 & 0.016339 & 0.018339 & 0.020279 \\ 
	0.001377 & 0.004122 & 0.006843 & 0.009524 & 0.012150 & 0.893286 & 0.017194 & 0.019593 & 0.021900 & 0.024112 \\ 
	0.001562 & 0.004672 & 0.007743 & 0.010753 & 0.013679 & 0.016504 & 0.862528 & 0.021804 & 0.024263 & 0.026591 \\ 
	0.001802 & 0.005387 & 0.008913 & 0.012345 & 0.015653 & 0.018814 & 0.021813 & 0.868310 & 0.027292 & 0.029770 \\ 
	0.001966 & 0.005870 & 0.009691 & 0.013384 & 0.016909 & 0.020241 & 0.023363 & 0.026268 & 0.840967 & 0.031441 \\ 
	0.002203 & 0.006571 & 0.010825 & 0.014902 & 0.018755 & 0.022352 & 0.025681 & 0.028739 & 0.031536 & 0.848536 \\ 
	\end{bmatrix}
}
\end{equation*}


Матрица $B^{-1}$:
\begin{equation*}
\scalemath{0.65}
{
	B^{-1} = 
	\begin{bmatrix}{}
	1.024264 & -0.000370 & -0.000647 & -0.000913 & -0.001229 & -0.001511 & -0.001866 & -0.002160 & -0.002553 & -0.002856 \\ 
	-0.000367 & 1.031055 & -0.001942 & -0.002738 & -0.003685 & -0.004525 & -0.005584 & -0.006457 & -0.007622 & -0.008514 \\ 
	-0.000613 & -0.001854 & 1.077996 & -0.004561 & -0.006131 & -0.007516 & -0.009259 & -0.010683 & -0.012581 & -0.014016 \\ 
	-0.000860 & -0.002600 & -0.004537 & 1.080458 & -0.008560 & -0.010470 & -0.012864 & -0.014797 & -0.017367 & -0.019276 \\ 
	-0.001109 & -0.003350 & -0.005838 & -0.008193 & 1.124533 & -0.013374 & -0.016375 & -0.018763 & -0.021929 & -0.024230 \\ 
	-0.001360 & -0.004103 & -0.007140 & -0.009998 & -0.013342 & 1.121993 & -0.019771 & -0.022553 & -0.026232 & -0.028837 \\ 
	-0.001612 & -0.004861 & -0.008443 & -0.011791 & -0.015680 & -0.018978 & 1.162763 & -0.026148 & -0.030254 & -0.033080 \\ 
	-0.001867 & -0.005623 & -0.009745 & -0.013568 & -0.017975 & -0.021657 & -0.026159 & 1.155762 & -0.033987 & -0.036957 \\ 
	-0.002124 & -0.006388 & -0.011046 & -0.015327 & -0.020219 & -0.024244 & -0.029131 & -0.032712 & 1.194080 & -0.040482 \\ 
	-0.002383 & -0.007157 & -0.012343 & -0.017063 & -0.022408 & -0.026733 & -0.031948 & -0.035677 & -0.040604 & 1.184143 \\ 
	\end{bmatrix}
}
\end{equation*}



\begin{equation*}
	\prescript{>}{}{x} = B^{-1} \cdot 
	\begin{bmatrix}{}
	1.000000 \\ 
	1.000000 \\ 
	1.000000 \\ 
	1.000000 \\ 
	1.000000 \\ 
	1.000000 \\ 
	1.000000 \\ 
	1.000000 \\ 
	1.000000 \\ 
	1.000000 \\ 
	\end{bmatrix}
	= 
	\begin{bmatrix}{}
	1.010158 \\ 
	0.989620 \\ 
	1.010780 \\ 
	0.989125 \\ 
	1.011372 \\ 
	0.988657 \\ 
	1.011916 \\ 
	0.988223 \\ 
	1.012407 \\ 
	0.987828 \\ 
	\end{bmatrix}
\end{equation*}

\begin{equation*}
	\prescript{>}{}{\Delta x} = 
	\begin{bmatrix}{}
	-0.010158 \\ 
	0.010380 \\ 
	-0.010780 \\ 
	0.010875 \\ 
	-0.011372 \\ 
	0.011343 \\ 
	-0.011916 \\ 
	0.011777 \\ 
	-0.012407 \\ 
	0.012172 \\ 
	\end{bmatrix},
	\qquad ||\prescript{>}{}{\Delta x}|| = 0.01240714
\end{equation*}

\textbf{Результаты.}
Число обусловленности $cond(A) = ||A|| \cdot ||A^{-1}|| = 1.406456 < 10^2$, значит матрица СЛАУ хорошо обусловлена. Следствием этого является малая относительная погрешность при решении приближенной СЛАУ:\\
 $\frac{|| \prescript{>}{}{\Delta x} ||}{|| \prescript{>}{}{x} ||} = 0.01240714 \leq cond(E + \lambda A) \frac{||\prescript{>}{}{\Delta b}||}{||\prescript{>}{}{b}||} = 1.406456 \cdot 0.01 = 0.01406456$
Кроме того, при делении каждого $i-$того уравнения $i=\overline{1,10}$ исходной СЛАУ на число $b^i + \Delta b^i$, абсолютная погрешность не изменилась: $||\prescript{>}{}{\Delta x}|| = 0.01240714$


\section*{Код программы}
Лабораторная работа выполнялась в среде программирования $R$ (R version 3.5.1 (2018-07-02) -- "Feather Spray"). Ниже приведён полный код программы, включающий экспорт таблиц в формат, нужный для вставки в \LaTeX:

\begin{lstlisting}
N <- 9
n <- 53
alpha <- (n - 50) / 100
A <- matrix(c(50 * (1 + 0.5 * N + alpha),
50 * (1 + 0.5 * N),
50 * (1 + 0.5 * N),
50.1 * (1 + 0.5 * N),
49.9 * (1 + 0.5 * N + alpha),
50 * (1 + 0.5 * N),
49.9 * (1 + 0.5 * N),
50 * (1 + 0.5 * N),
50.1 * (1 + 0.5 * N + alpha)),
nrow = 3, ncol = 3, byrow = T)
A
B <- matrix(c(50 * (3 + 1.5* N + alpha),
50 * (3 + 1.5* N + alpha),
50 * (3 + 1.5* N + alpha)))
B
B_dB <- matrix(c(50 * (1 + 0.01) * (3 + 1.5* N + alpha),
50 * (1 - 0.01) * (3 + 1.5* N + alpha),
50 * (1 + 0.01) * (3 + 1.5* N + alpha)))
B_dB

A_inv <- solve(A)

A_inv_norm <- norm(A_inv, "i")
A_norm <- norm(A, "i")

cond_A <- A_inv_norm * A_norm
cond_A
library(matlib)
library(xtable)
showEqn(round(A, 4), B, latex=TRUE)
showEqn(round(A, 4), B_dB, latex=TRUE)

x <-xtable(A,align=rep("",ncol(A)+1), digits = 3)
print(x, floating=FALSE, tabular.environment="bmatrix", 
hline.after=NULL, include.rownames=FALSE, include.colnames=FALSE)

x <-xtable(B,align=rep("",ncol(B)+1), digits = 3)
print(x, floating=FALSE, tabular.environment="bmatrix", 
hline.after=NULL, include.rownames=FALSE, include.colnames=FALSE)

x <-xtable(A_inv,align=rep("",ncol(A_inv)+1), digits = 3)
print(x, floating=FALSE, tabular.environment="bmatrix", 
hline.after=NULL, include.rownames=FALSE, include.colnames=FALSE)

x <-xtable(B_dB,align=rep("",ncol(B_dB)+1), digits = 3)
print(x, floating=FALSE, tabular.environment="bmatrix", 
hline.after=NULL, include.rownames=FALSE, include.colnames=FALSE)
###################################################################################
#
X <- solve(A, B)


x <-xtable(X,align=rep("",ncol(X)+1), digits = 7)
print(x, floating=FALSE, tabular.environment="bmatrix", 
hline.after=NULL, include.rownames=FALSE, include.colnames=FALSE)

A %*% X

delta_B <- B_dB - B
delta_B

delta_B_norm <- norm(delta_B, "i")
delta_B_norm

B_norm <- norm(B, "i")
B_norm

delta_X <- solve(A, delta_B)
delta_X

x <-xtable(delta_X,align=rep("",ncol(delta_X)+1), digits = 7)
print(x, floating=FALSE, tabular.environment="bmatrix", 
hline.after=NULL, include.rownames=FALSE, include.colnames=FALSE)

X_norm <- norm(X, "i")
X_norm

delta_X_norm <- norm(delta_X, "i")
delta_X_norm

(delta_X_norm / X_norm) <= cond_A * (delta_B_norm / B_norm)

######################################################################################
rm(list = ls())
N <- 9
n <- 53
alpha <- (n - 50) / 100
lambda <- 0.6 - alpha
f <- function(x) atan(x)
a <- 0
b <- 1
h <- (b - a) / 10

s1 <- a + h / 2
s2 <- s1 + h
s3 <- s2 + h
s4 <- s3 + h
s5 <- s4 + h
s6 <- s5 + h
s7 <- s6 + h
s8 <- s7 + h
s9 <- s8 + h
s10 <- s9 + h

s <- c(s1, s2, s3, s4, s5, s6, s7, s8, s9, s10)
t <- s
s[2]

a_ij <- matrix(rep(NA, 100), nrow = 10, ncol = 10)
for (i in 1:10)
{
for (j in 1:10)
{
a_ij[i,j] <- f(s[i] * t[j]) * h
}
}

x <-xtable(a_ij,align=rep("",ncol(a_ij)+1), digits = 7)
print(x, floating=FALSE, tabular.environment="bmatrix", 
hline.after=NULL, include.rownames=FALSE, include.colnames=FALSE)

e_plus_lambda_a <- diag(10) + lambda * a_ij # E + lambda*A
e_plus_lambda_a

x <-xtable(e_plus_lambda_a,align=rep("",ncol(e_plus_lambda_a)+1), digits = 7)
print(x, floating=FALSE, tabular.environment="bmatrix", 
hline.after=NULL, include.rownames=FALSE, include.colnames=FALSE)

e_plus_lambda_a_inv <- solve(e_plus_lambda_a)
e_plus_lambda_a_inv

x <-xtable(e_plus_lambda_a_inv,align=rep("",ncol(e_plus_lambda_a_inv)+1), digits = 6)
print(x, floating=FALSE, tabular.environment="bmatrix", 
hline.after=NULL, include.rownames=FALSE, include.colnames=FALSE)

e_plus_lambda_a_norm <- norm(e_plus_lambda_a, "i")
e_plus_lambda_a_norm

e_plus_lambda_a_inv_norm <- norm(e_plus_lambda_a_inv, "i")
e_plus_lambda_a_inv_norm

cond_e_pl_lm <- e_plus_lambda_a_norm * e_plus_lambda_a_inv_norm
cond_e_pl_lm

X <- matrix(rep(1,10), ncol = 1)
B <- e_plus_lambda_a %*% X
B_norm <- norm(B, "i")

x <-xtable(B,align=rep("",ncol(B)+1), digits = 6)
print(x, floating=FALSE, tabular.environment="bmatrix", 
hline.after=NULL, include.rownames=FALSE, include.colnames=FALSE)


delta_B <- 0.01 * as.matrix(ifelse(seq(1,10,1) %% 2, B, -B))
delta_B

x <-xtable(delta_B,align=rep("",ncol(delta_B)+1), digits = 6)
print(x, floating=FALSE, tabular.environment="bmatrix", 
hline.after=NULL, include.rownames=FALSE, include.colnames=FALSE)

delta_B_norm <- norm(delta_B, "i")
delta_B_norm

B_pl_delta_B <- B + delta_B
B_pl_delta_B

x <-xtable(B_pl_delta_B,align=rep("",ncol(B_pl_delta_B)+1), digits = 6)
print(x, floating=FALSE, tabular.environment="bmatrix", 
hline.after=NULL, include.rownames=FALSE, include.colnames=FALSE)

B_pl_delta_B_norm <- norm(B_pl_delta_B, "i")
B_pl_delta_B_norm

X_plus_delta_X <- solve(e_plus_lambda_a, B + delta_B)
X_plus_delta_X

x <-xtable(X_plus_delta_X,align=rep("",ncol(X_plus_delta_X)+1), digits = 6)
print(x, floating=FALSE, tabular.environment="bmatrix", 
hline.after=NULL, include.rownames=FALSE, include.colnames=FALSE)

delta_X <- X_plus_delta_X - 1
delta_X

x <-xtable(delta_X,align=rep("",ncol(delta_X)+1), digits = 6)
print(x, floating=FALSE, tabular.environment="bmatrix", 
hline.after=NULL, include.rownames=FALSE, include.colnames=FALSE)

X_approx <- X_plus_delta_X - delta_X
X_approx

delta_X_norm <- norm(delta_X, "i")
delta_X_norm
X_approx_norm <- norm(X_approx, "i")
X_approx_norm

relative_error <- delta_X_norm / X_approx_norm
relative_error
cond_e_pl_lm * (delta_B_norm / B_norm)


temp <- B + delta_B

A_divided2 <- matrix(rep(NA, 100), nrow = 10, ncol = 10)

for (i in 1:10)
{
for (j in 1:10)
{
A_divided2[i,j] <- e_plus_lambda_a[i,j] / temp[i]
}
}
A_divided2

x <-xtable(A_divided2,align=rep("",ncol(A_divided2)+1), digits = 6)
print(x, floating=FALSE, tabular.environment="bmatrix", 
hline.after=NULL, include.rownames=FALSE, include.colnames=FALSE)


A_divided2_inv <- solve(A_divided2)

x <-xtable(A_divided2_inv,align=rep("",ncol(A_divided2_inv)+1), digits = 6)
print(x, floating=FALSE, tabular.environment="bmatrix", 
hline.after=NULL, include.rownames=FALSE, include.colnames=FALSE)


X_sec <- A_divided2_inv %*% matrix(rep(1,10), ncol = 1)
X_sec

x <-xtable(matrix(rep(1,10), ncol = 1),align=rep("",ncol(matrix(rep(1,10), ncol = 1))+1), digits = 6)
print(x, floating=FALSE, tabular.environment="bmatrix", 
hline.after=NULL, include.rownames=FALSE, include.colnames=FALSE)

x <-xtable(X_sec,align=rep("",ncol(X_sec)+1), digits = 6)
print(x, floating=FALSE, tabular.environment="bmatrix", 
hline.after=NULL, include.rownames=FALSE, include.colnames=FALSE)

delta_X_sec <- 1 - X_sec
delta_X_sec

x <-xtable(delta_X_sec,align=rep("",ncol(delta_X_sec)+1), digits = 6)
print(x, floating=FALSE, tabular.environment="bmatrix", 
hline.after=NULL, include.rownames=FALSE, include.colnames=FALSE)

delta_X_sec_norm <- norm(delta_X_sec, "i")
delta_X_sec_norm

\end{lstlisting}



\end{document}