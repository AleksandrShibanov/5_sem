\documentclass[14pt,a4paper]{scrartcl}
\usepackage{cmap}
\usepackage[utf8]{inputenc}
\usepackage[T1,T2A]{fontenc}
\usepackage[english,russian]{babel}
\usepackage{relsize}
\usepackage{graphicx}
\usepackage{subfigure}
\usepackage{mathtools}
\usepackage{amssymb}
\usepackage{float}
\usepackage{sidecap}
\usepackage{wrapfig}
\usepackage{caption}
\usepackage[table,xcdraw]{xcolor}
\usepackage{listings}
\usepackage{amsmath,cryptocode}
\usepackage{listings}
\usepackage{booktabs}
\usepackage{multirow}  
\usepackage{multicol}
\usepackage{bigstrut}
\usepackage{lscape}
\usepackage{rotating}
\usepackage{adjustbox}
\usepackage{minted}


\newcommand\scalemath[2]{\scalebox{#1}{\mbox{\ensuremath{\displaystyle #2}}}}


\begin{document}
	\begin{titlepage}
	\begin{center}
		\large
		МИНИСТЕРСТВО ОБРАЗОВАНИЯ И НАУКИ\\ РОССИЙСКОЙ ФЕДЕРАЦИИ
		
		\vspace{0.5cm}
		
		МГТУ им Н.Э.Баумана
		\vspace{0.25cm}
		
		Факультет ФН
		
		Кафедра вычислительной математики и математической физики
		\vfill
		
		
		Соколов Арсений Андреевич\\
		\vfill
		
		
		{\LARGE Лабораторная работа №4 по численным методам\\[2mm]
		}
		\bigskip
		
		3 курс, группа ФН11-53Б\\
		Вариант 6
	\end{center}
	\vfill
	
	\newlength{\ML}
	\settowidth{\ML}{«\underline{\hspace{0.7cm}}» \underline{\hspace{2cm}}}
	\hfill\begin{minipage}{0.4\textwidth}
		Преподаватель\\
		\underline{\hspace{3cm}} В.\,А.~Кутыркин\\
		«\underline{\hspace{0.7cm}}» \underline{\hspace{1.71cm}} 2019 г.
	\end{minipage}%
	\bigskip
	
	
	\vfill
	
	\begin{center}
		Москва, 2019 г.
	\end{center}
\end{titlepage}

\section*{Задание 1}
\textbf{Задание.}\\
Используя метод Якоби, найти приближённое решение полной спектральной задачи для матрицы A, приведённой в таблицах ниже. Останов выбрать на том шаге итерации, когда максимальный по модулю внедиагональный элемент преобразованной матрицы станет меньше $\epsilon = $ 0.01. Проверить найденные приближённые собственные векторы и отвечающие им собственные значения матрицы A.\\
\textbf{Исходные данные.}\\
$N = 6, n = 53$

$$ A =
\begin{pmatrix}
	12 & 1  & 2  & 3  \\
	1  & 12 & 3  & -2 \\
	2  & 3  & 12 & 1  \\
	3  & -2  & 1  & 12
\end{pmatrix}
$$

\textbf{Решение.}\\
Согласано методу Якоби будем составлять матрицы $Q$ и $A$ соответствующим образом:
\begin{equation*}
	\boldsymbol{Q}(\alpha, \beta ; \varphi)=\left\langle\boldsymbol{e}_{1}, \ldots, \boldsymbol{e}_{\alpha-1},^{>} \boldsymbol{s}(\alpha, \beta ; \varphi), \boldsymbol{e}_{\alpha+1}, \ldots, \boldsymbol{e}_{\beta-1},^{>} \boldsymbol{t}(\alpha, \beta ; \varphi), \boldsymbol{e}_{\beta+1}, \ldots, \boldsymbol{e}_{n}\right],
\end{equation*}
где
\begin{equation*}
	\left\{\begin{array}{l}{\boldsymbol{s}(\alpha, \beta ; \varphi)=\sum_{k=1}^{n}\left(\delta_{\alpha}^{k} \cos \varphi+\delta_{\beta}^{k} \sin \varphi\right) \boldsymbol{e}_{k}} \\ {\boldsymbol{t}(\alpha, \beta ; \varphi)=\sum_{k=1}^{n}\left(-\delta_{\alpha}^{k} \sin \varphi+\delta_{\beta}^{k} \cos \varphi\right) \boldsymbol{e}_{k}}\end{array}\right.,
\end{equation*}
где
\begin{equation*}
	\varphi = \frac{1}{2} arcctg \frac{a_\alpha^\alpha[k] - a_\beta^\beta[k]}{2a_\alpha^\beta}
\end{equation*}

На $i$-ой итерации будем искать матрицу A как
\begin{equation*}
	A_i = \prescript{T}{}{Q_{i-1}} \cdot A_{i-1} \cdot Q_{i-1},
\end{equation*}
для $i={1,2,\ldots}$, где $A_0 = A, Q_0 = Q$.

\textit{Итерация 1.}\\
$$\alpha = 1, \beta = 4$$

$$\varphi = 0.7853982 $$

$$ Q =
\begin{pmatrix}
0.7071068 & 0 & 0 & -0.7071068 \\
0.0000000 & 1 & 0 & 0.0000000  \\
0.0000000 & 0 & 1 & 0.0000000  \\
0.7071068 & 0 & 0 & 0.7071068 
\end{pmatrix}
$$
$$
A_1 = 
\begin{pmatrix}
15.00000 & -0.70711 & 2.12132 & 0.00000 \\ 
-0.70711 & 12.00000 & 3.00000 & -2.12132 \\ 
2.12132 & 3.00000 & 12.00000 & -0.70711 \\ 
0.00000 & -2.12132 & -0.70711 & 9.00000 \\ 
\end{pmatrix}
$$
%#############################################################################
\textit{Итерация 2.}\\
$$\alpha = 2, \beta = 3$$

$$\varphi_1 = 0.7853982 $$

$$ Q_1 =
\begin{pmatrix}
1.00000 & 0.00000 & 0.00000 & 0.00000 \\ 
0.00000 & 0.70711 & -0.70711 & 0.00000 \\ 
0.00000 & 0.70711 & 0.70711 & 0.00000 \\ 
0.00000 & 0.00000 & 0.00000 & 1.00000 \\ 
\end{pmatrix}
$$

$$
A_2 = 
\begin{pmatrix}
15.00000 & 1.00000 & 2.00000 & 0.00000 \\ 
1.00000 & 15.00000 & 0.00000 & -2.00000 \\ 
2.00000 & 0.00000 & 9.00000 & 1.00000 \\ 
0.00000 & -2.00000 & 1.00000 & 9.00000 \\ 
\end{pmatrix}
$$

%##############################################################################
\textit{Итерация 3.}\\
$$\alpha = 1, \beta = 3$$

$$\varphi_2 = 0.2940013 $$

$$ Q_2 =
\begin{pmatrix}
0.95709 & 0.00000 & -0.28978 & 0.00000 \\ 
0.00000 & 1.00000 & 0.00000 & 0.00000 \\ 
0.28978 & 0.00000 & 0.95709 & 0.00000 \\ 
0.00000 & 0.00000 & 0.00000 & 1.00000 \\ 
\end{pmatrix}
$$

$$
A_3 = 
\begin{pmatrix}
15.60555 & 0.95709 & 0.00000 & 0.28978 \\ 
0.95709 & 15.00000 & -0.28978 & -2.00000 \\ 
0.00000 & -0.28978 & 8.39445 & 0.95709 \\ 
0.28978 & -2.00000 & 0.95709 & 9.00000 \\ 
\end{pmatrix}
$$
%##############################################################################



\textit{Итерация 4.}\\
$$\alpha = 2, \beta = 4$$

$$\varphi_3 = -0.2940013 $$

$$ Q_3 =
\begin{pmatrix}
1.00000 & 0.00000 & 0.00000 & 0.00000 \\ 
0.00000 & 0.95709 & 0.00000 & 0.28978 \\ 
0.00000 & 0.00000 & 1.00000 & 0.00000 \\ 
0.00000 & -0.28978 & 0.00000 & 0.95709 \\ 
\end{pmatrix}
$$

$$
A_4 = 
\begin{pmatrix}
15.60555 & 0.83205 & 0.00000 & 0.55470 \\ 
0.83205 & 15.60555 & -0.55470 & 0.00000 \\ 
0.00000 & -0.55470 & 8.39445 & 0.83205 \\ 
0.55470 & 0.00000 & 0.83205 & 8.39445 \\  
\end{pmatrix}
$$

%##############################################################################



\textit{Итерация 5.}\\
$$\alpha = 3, \beta = 4$$

$$\varphi_4 = 0.7853982 $$

$$ Q_4 =
\begin{pmatrix}
1.00000 & 0.00000 & 0.00000 & 0.00000 \\ 
0.00000 & 1.00000 & 0.00000 & 0.00000 \\ 
0.00000 & 0.00000 & 0.70711 & -0.70711 \\ 
0.00000 & 0.00000 & 0.70711 & 0.70711 \\ 
\end{pmatrix}
$$

$$
A_5 = 
\begin{pmatrix}
15.60555 & 0.83205 & 0.39223 & 0.39223 \\ 
0.83205 & 15.60555 & -0.39223 & 0.39223 \\ 
0.39223 & -0.39223 & 9.22650 & 0.00000 \\ 
0.39223 & 0.39223 & 0.00000 & 7.56240 \\ 
\end{pmatrix}
$$

%##############################################################################



\textit{Итерация 6.}\\
$$\alpha = 1, \beta = 2$$

$$\varphi_5 = 0.7853982 $$

$$ Q_5 =
\begin{pmatrix}
0.70711 & -0.70711 & 0.00000 & 0.00000 \\ 
0.70711 & 0.70711 & 0.00000 & 0.00000 \\ 
0.00000 & 0.00000 & 1.00000 & 0.00000 \\ 
0.00000 & 0.00000 & 0.00000 & 1.00000 \\ 
\end{pmatrix}
$$

$$
A_6 = 
\begin{pmatrix}
16.43760 & 0.00000 & 0.00000 & 0.55470 \\ 
0.00000 & 14.77350 & -0.55470 & -0.00000 \\ 
0.00000 & -0.55470 & 9.22650 & 0.00000 \\ 
0.55470 & -0.00000 & 0.00000 & 7.56240 \\ 
\end{pmatrix}
$$

%##############################################################################



\textit{Итерация 7.}\\
$$\alpha = 1, \beta = 4$$

$$\varphi_6 = 0.0621775 $$

$$ Q_6 =
\begin{pmatrix}
0.99807 & 0.00000 & 0.00000 & -0.06214 \\ 
0.00000 & 1.00000 & 0.00000 & 0.00000 \\ 
0.00000 & 0.00000 & 1.00000 & 0.00000 \\ 
0.06214 & 0.00000 & 0.00000 & 0.99807 \\ 
\end{pmatrix}
$$

$$
A_7 = 
\begin{pmatrix}
16.47214 & -0.00000 & 0.00000 & 0.00000 \\ 
-0.00000 & 14.77350 & -0.55470 & -0.00000 \\ 
0.00000 & -0.55470 & 9.22650 & -0.00000 \\ 
0.00000 & -0.00000 & -0.00000 & 7.52786 \\ 
\end{pmatrix}
$$

%##############################################################################



\textit{Итерация 8.}\\
$$\alpha = 2, \beta = 3$$

$$\varphi_7 = -0.09869778 $$

$$ Q_7 =
\begin{pmatrix}
1.00000 & 0.00000 & 0.00000 & 0.00000 \\ 
0.00000 & 0.99513 & 0.09854 & 0.00000 \\ 
0.00000 & -0.09854 & 0.99513 & 0.00000 \\ 
0.00000 & 0.00000 & 0.00000 & 1.00000 \\ 
\end{pmatrix}
$$

$$
A_8 = 
\begin{pmatrix}
16.47214 & -0.00000 & 0.00000 & 0.00000 \\ 
-0.00000 & 14.82843 & 0.00000 & -0.00000 \\ 
0.00000 & 0.00000 & 9.17157 & -0.00000 \\ 
0.00000 & -0.00000 & -0.00000 & 7.52786 \\ 
\end{pmatrix}
$$

Получили, что на восьмой итерации достигнута заданная точность.\\
Проверим полученные собственные значения с помощью функции $\boldmath{eigen}$, встроенной в пакет $R$, в котором выполнялась данная работа:
\begin{minted}{R}
> print(eigen(A)$values, digits = 6)
[1] 16.47214 14.82843  9.17157  7.52786
\end{minted}

Получили, что решения совпадают.

Соответствующие собственные вектора:
$$
v_1 = 
\begin{pmatrix}
	-0.601501\\ -0.371748\\ -0.601501\\ -0.371748\\
\end{pmatrix};
v_2 = 
\begin{pmatrix}
	0.2705981\\ -0.6532815\\ -0.2705981\\  0.6532815\\
\end{pmatrix};
v_3 = 
\begin{pmatrix}
	 0.6532815\\  0.2705981\\ -0.6532815\\ -0.2705981\\
\end{pmatrix};
v_4 = 
\begin{pmatrix}
	-0.371748\\  0.601501\\ -0.371748\\  0.601501\\
\end{pmatrix}.
$$

















\end{document}