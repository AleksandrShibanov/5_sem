\documentclass[14pt,a4paper]{scrartcl}
\usepackage{cmap}
\usepackage[utf8]{inputenc}
\usepackage[T1,T2A]{fontenc}
\usepackage[english,russian]{babel}
\usepackage{relsize}
\usepackage{graphicx}
\usepackage{subfigure}
\usepackage{mathtools}
\usepackage{amssymb}
\usepackage{float}
\usepackage{sidecap}
\usepackage{wrapfig}
\usepackage{caption}
\usepackage[table,xcdraw]{xcolor}
\usepackage{listings}
\begin{document}
	\begin{titlepage}
	\begin{center}
		\large
		МИНИСТЕРСТВО ОБРАЗОВАНИЯ И НАУКИ\\ РОССИЙСКОЙ ФЕДЕРАЦИИ
		
		\vspace{0.5cm}
		
		МГТУ им Н.Э.Баумана
		\vspace{0.25cm}
		
		Факультет ФН
		
		Кафедра вычислительной математики и математической физики
		\vfill
		
		
		Соколов Арсений Андреевич\\
		\vfill
		
		
		{\LARGE Домашнее задание №1 по математической статистике\\[2mm]
		}
		\bigskip
		
		3 курс, группа ФН11-53Б\\
		Вариант 9
	\end{center}
	\vfill
	
	\newlength{\ML}
	\settowidth{\ML}{«\underline{\hspace{0.7cm}}» \underline{\hspace{2cm}}}
	\hfill\begin{minipage}{0.4\textwidth}
		Преподаватель\\
		\underline{\hspace{3cm}} Т.\,В.~Облакова\\
		«\underline{\hspace{0.7cm}}» \underline{\hspace{1.71cm}} 2019 г.
	\end{minipage}%
	\bigskip
	
	
	\vfill
	
	\begin{center}
		Москва, 2019 г.
	\end{center}
\end{titlepage}

\section{Подготовка данных}
\subsection{Поиск крайних элементов вариационного ряда}
Импортируем данные, сохранённые в формате $.csv$, определяем максимальный, минимальный.


\begin{lstlisting}
	> df <- read.csv("db.csv", header = F) #data import
	> max_el <- max(df)
	> max_el
	[1] 17.985
	> min_el <- min(df)
	> min_el
	[1] 2.375
	> range_el <- max_el - min_el
	> range_el
	[1] 15.61
\end{lstlisting}
\subsection{Вариационный ряд}
Выведем на печать упорядоченную выборку -- вариационный ряд:
\begin{lstlisting}
> sort(df$V1)
[1]  2.375  2.666  2.744  2.867  2.903  3.086  3.375
[8]  3.377  3.608  3.635  3.878  4.354  4.646  4.715
[15]  4.727  4.843  4.942  4.964  5.006  5.140  5.175
[22]  5.333  5.343  5.647  5.743  5.811  5.814  6.078
[29]  6.339  6.355  6.542  6.734  6.875  6.883  7.039
[36]  7.175  7.241  7.357  7.479  7.683  8.206  8.250
[43]  8.282  8.415  8.428  8.592  8.681  8.905  8.916
[50]  9.012  9.097  9.159  9.182  9.234  9.354  9.458
[57]  9.493  9.838  9.861  9.869  9.919  9.927 10.268
[64] 10.351 10.511 10.821 11.093 11.095 11.449 12.040
[71] 12.173 12.175 12.177 12.423 12.622 12.767 12.768
[78] 12.888 12.906 12.911 12.924 12.959 12.959 13.325
[85] 13.480 13.535 13.890 13.924 13.957 14.022 14.186
[92] 14.237 14.286 14.396 14.495 14.612 14.919 15.134
[99] 15.238 15.308 15.439 15.506 15.535 15.649 15.694
[106] 15.850 15.864 15.988 15.992 16.058 16.175 16.195
[113] 16.892 17.103 17.117 17.690 17.757 17.921 17.963
[120] 17.985
\end{lstlisting}

\section{Построение гистограммы частот}
\subsection{Определение количества интервалов разбиения}
Определим число "столбиков" гистограммы по правилу Стёрджеса:
\begin{equation*}
	n = 1 + [\log_2{N}],
\end{equation*}
где $N$ -- общее число наблюдений.


\begin{lstlisting}
> num_bins = round(1 + log2(length(df$V1)))
> num_bins
[1] 8
\end{lstlisting}

\subsection{Определение ширины интервала}
Определим ширину интервала группировки по формуле:
\begin{equation*}
	\text{width} = \frac{\text{range}}{n},
\end{equation*}
где $\text{range}$ -- размах выборки.


\begin{lstlisting}
> bin_width <- range_el / num_bins
> bin_width
[1] 1.95125
\end{lstlisting}


\subsection{Построение графика}
По полученным данным построим гистограмму частот:


\begin{lstlisting}
> png(filename = "../img/hist_without_dens.png", 
+     width = 1920, height = 1080,
+     pointsize = 24, res = 96 * 1.25)
> par(mar = c(3, 3, 2, 1), xaxs = "i", yaxs = "i")
> pl1 <- hist(df$V1,
+        breaks = seq(min_el, max_el, by = bin_width), 
+        xlim = c(0, 20), ylim = c(0.00,0.10), 
+				axes = F, freq = F,
+        main = "Histogram of data")
> axis(1, seq(0, 20, 1))
> axis(2, seq(0.00, 0.10, 0.01), las = 1)
> grid(nx = 20, ny = 10, equilogs = F)
> dev.off()
\end{lstlisting}

\begin{figure}[t!]
	\center{\includegraphics[width=1\linewidth]{../img/hist_without_dens.png}}
	\caption{Гистограмма частот выборки.}
	\label{ris:hist_without_dens}
\end{figure}

\subsection{Определение относительных частот}
Для начала определим количество элементов в каждом интервале:
\begin{lstlisting}
> pl1$counts
[1] 11 17 13 21  8 20 19 11
\end{lstlisting}

Посчитаем относительную частоту (эмпирическую вероятность) для каждого интервала:
\begin{lstlisting}
> relative_freq <- pl1$counts / length(df$V1)
> relative_freq
[1] 0.09166667 0.14166667 0.10833333 0.17500000
[5] 0.06666667 0.16666667 0.15833333 0.09166667
\end{lstlisting}

\subsection{Меры центральной тенденции}
Определим выборочные среднее и медиану (мода в нашей выборке будет неинформативной):
\begin{lstlisting}
> mean_el <- mean(df$V1)
> mean_el
[1] 10.2681
> median_el <- median(df$V1)
> median_el
[1] 9.894
\end{lstlisting}

\subsection{Меры изменчивости}
Определим размах(определялся ранее), дисперсию и стандартное отклонение:
\begin{lstlisting}
> range_el
[1] 15.61
> var_el <- var(df$V1)
> var_el
[1] 19.61091
> sd_el <- sd(df$V1)
> sd_el
[1] 4.42842
\end{lstlisting}


\section{Определение возможного закона распределения}
\subsection{Предварительный анализ}
Анализ возможного закона распределения начнём с построения Cullen and Frey skewness-kurtosis графика, предварительно забустрэпив $10^4$ ресэмплов:
\begin{lstlisting}
> png(filename = "../img/cullen_and_frey_graph.png", 
+     width = 1920, height = 1080,
+     pointsize = 24, res = 96 * 1.25)
> descdist(df$V1,method = "sample", boot = 1e4)
summary statistics
------
min:  2.375   max:  17.985 
median:  9.894 
mean:  10.2681 
sample sd:  4.40993 
sample skewness:  -0.02444883 
sample kurtosis:  1.822814 
> dev.off() 
\end{lstlisting}

Где $\text{kurtosis} = \frac{\mu_4}{\sigma^4} - 3$ -- это коэффициент эксцесса, а $\text{skewness} = \frac{\mu_3}{\sigma^3} $ -- коэффициент асимметрии.

\begin{figure}[t!]
	\center{\includegraphics[width=1\linewidth]{../img/cullen_and_frey_graph.png}}
	\caption{Cullen and Frey skewness-kurtosis plot.}
	\label{ris:cullen_and_frey_graph}
\end{figure}

\pagebreak
Из Рис.~\ref{ris:cullen_and_frey_graph} мы видим, что наша выборка близка больше всего к $\text{uniform}$ -- равномерному распределению и $\text{beta}$ -- бета-распределению. Так же рассмотрим нормальное распределение для сравнения моделей.

Для анализа нашего датасета на принадлежность к бета-распределению, зарескейлим данные к интервалу $[0;1]$:
\begin{lstlisting}
> library(scales)
> res <- rescale(df$V1)
> res
[1] 0.77642537 0.19013453 0.77008328 0.67347854
[5] 0.00000000 0.43939782 0.22011531 0.15067265
[9] 0.15810378 0.06418962 0.40397181 0.62767457
[13] 0.14990391 1.00000000 0.39827034 0.50563741
[17] 0.99590006 0.23721973 0.31172325 0.14548366
[21] 0.48379244 0.88532992 0.16854580 0.98539398
[25] 0.30749520 0.87232543 0.37354260 0.43606662
[29] 0.43062140 0.16585522 0.22030750 0.84304933
[33] 0.86412556 0.61915439 0.07898783 0.28878924
[37] 0.38776425 0.73766816 0.75989750 0.20960922
[41] 0.86322870 0.25496477 0.04554773 0.48327995
[45] 0.08071749 0.66579116 0.03151826 0.01864190
[49] 0.55848815 0.47809097 0.31915439 0.37841127
[53] 0.58129404 0.74196028 0.28827675 0.94439462
[57] 0.99859065 0.02363869 0.62793081 0.47956438
[61] 0.06406150 0.73984625 0.54106342 0.03382447
[65] 0.55861627 0.67495195 0.09628443 0.51095452
[69] 0.37636131 0.75663037 0.84119154 0.21575913
[73] 0.67463165 0.42517617 0.66572710 0.87206919
[77] 0.45598975 0.85323511 0.18949391 0.92998078
[81] 0.17713004 0.44708520 0.34003844 0.88404869
[85] 0.38693145 0.45374760 0.87655349 0.67802691
[89] 0.62780269 0.76303652 0.81736067 0.64368994
[93] 0.27924407 0.83689942 0.74612428 0.82850737
[97] 0.41902627 0.98110186 0.67802691 0.80358744
[101] 0.32696989 0.48007687 0.67578475 0.52120436
[105] 0.65643818 0.78392056 0.94349776 0.29878283
[109] 0.71140295 0.26694427 0.12677771 0.25393978
[113] 0.71492633 0.17937220 0.43459321 0.16444587
[117] 0.70147341 0.85035234 0.41832159 0.82402306
\end{lstlisting}

Составим наши модели, отталкиваясь от метода моментов:
\begin{lstlisting}
> fit.norm <- fitdist(df$V1, "norm", method = "mme")
> fit.uniform <- fitdist(df$V1, "unif", method = "mme")
> fit.beta <- fitdist(res, "beta", method = "mme")
\end{lstlisting}

Рассмотрим критерии Колмогорова-Смирнова и Крамера-Мизеса:
\begin{lstlisting}
> stat_normal <- gofstat(fit.norm, 
+                        fitnames = c("normal"))
> stat_unif <- gofstat(fit.uniform, 
+                      fitnames = c("uniform"))
> stat_beta <- gofstat(fit.beta, 
+                      fitnames = c("beta"))
> 
> stat_table <- cbind(rbind(stat_normal$ks, stat_normal$cvm),  
+                     rbind(stat_unif$ks, stat_unif$cvm),
+                     rbind(stat_beta$ks, stat_beta$cvm))
> rownames(stat_table) <- c("Kolmogorov-Smirnov statistic", 
+                           "Cramer-von Mises statistic")
> stat_table
                               normal    uniform     beta
Kolmogorov-Smirnov statistic 0.08952502 0.04535694 0.04307507
Cramer-von Mises statistic   0.22184007 0.03095026 0.02950918
\end{lstlisting}

Отчётливо видно преобладание равномерного и бета-распределений с небольшим преимуществом у бета-распределения.

Построим несколько графиков:
\begin{lstlisting}
> png(filename = "../img/fit_norm.png", 
+     width = 1920, height = 1080,
+     pointsize = 24, res = 96 * 1)
> plot(fit.norm)
> dev.off()
> png(filename = "../img/fit_unif.png", 
+     width = 1920, height = 1080,
+     pointsize = 24, res = 96 * 1)
> plot(fit.uniform)
> dev.off() 
> png(filename = "../img/fit_beta.png", 
+     width = 1920, height = 1080,
+     pointsize = 24, res = 96 * 1)
> plot(fit.beta)
> dev.off()
\end{lstlisting}

\begin{figure}[h]
	\center{\includegraphics[width=1\linewidth]{../img/fit_norm.png}}
	\caption{Нормальное распределение.}
	\label{ris:fit_norm}
\end{figure}

\begin{figure}[t!]
	\center{\includegraphics[width=1\linewidth]{../img/fit_unif.png}}
	\caption{Равномерное распределение.}
	\label{ris:fit_unif}
\end{figure}

\begin{figure}[t!]
	\center{\includegraphics[width=1\linewidth]{../img/fit_beta.png}}
	\caption{Бета-распределение.}
	\label{ris:fit_beta}
\end{figure}




\newpage

Для простоты дальнейшего анализа рассмотрим равномерное распределение.
\subsection{Оценка параметров распределения методом моментов}
Для оценки параметров методом моментов приравняем выборочные моменты распределения (среднее значение и несмещенную дисперсию) к теоретическим:
\begin{equation*}
	\hat{\mu_1} = \frac{1}{N}\sum_{k=1}^{N}x_k = \mathbb{E}[X]
\end{equation*}
\begin{equation*}
	\hat{\mu_2} = \frac{1}{N-1}\sum_{k=1}^{N}(x_k-\bar{x})^2 = D[X]
\end{equation*}
Рассмотрим равномерное распределение. Его плотность:
\begin{equation*}
	f_{X}(x)=\left\{\begin{array}{ll}{\lambda e^{-\lambda x},} & {x \geq 0} \\ {0,} & {x<0}\end{array}\right.
\end{equation*}
И соответствующие моменты:
\begin{equation*}
	\mathbb{E}[X] = \frac{a+b}{2}
\end{equation*}
\begin{equation*}
	D[X] = \frac{(b-a)^2}{12}
\end{equation*}
Тогда составим систему: \\
\begin{equation*}
	\begin{cases*}
		\mathbb{E}[X] = \frac{a+b}{2} = 10.2681 \\
		D[X] = \frac{(b-a)^2}{12} = 19.6109
	\end{cases*}
\end{equation*}

Мы видим, что полученная система состоит из двух симметрических многочленов, поэтому будет иметь место два симметричных решения, из которых следует выбрать то решение, где $b > a$: \\
\begin{equation*}
	\begin{cases*}
		a=2.597 \\
		b=17.938 \\
	\end{cases*}
\end{equation*}

Тогда исходная случайная величина имеет непрерывное равномерное распределение на отрезке $[a;b] = [2.597; 17.938]$ и ее плотность $f_X(x)$ имеет вид:

\begin{equation*}
	f_X(x) = 
	\begin{cases*}
		\frac{1}{15.341}, & x $\in$ [2.597; 17.938]  \\
		0, & x $\notin$ [2.597; 17.938]. \\
	\end{cases*}
\end{equation*}

\subsection{Построение совмещённого графика гистограммы и плотности предполагаемого закона}
По полученным ранее данным, построим совмещённые графики плотности равномерного закона и нашей гистограммы:
\begin{lstlisting}
png(filename = "../img/hist_with_unif_dens.png", 
width = 1920, height = 1080,
pointsize = 24, res = 96 * 1.25)
par(mar = c(3, 3, 2, 1), xaxs = "i", yaxs = "i")
pl1 <- hist(df$V1,
breaks = seq(min_el, max_el, by = bin_width), 
xlim = c(0, 20), ylim = c(0.00,0.10), axes = F, freq = F,
main = "Histogram of data")
axis(1, seq(0, 20, 1))
axis(2, seq(0.00, 0.10, 0.01), las = 1)
grid(nx = 20, ny = 10, equilogs = F)
curve(dunif(x, 2.5978, 17.9383), 2.5978, 17.9383, 
xlim = c(0,20), add = T, col = "red", lwd = 3)
legend("topright", c("uniform density"), 
lty=c(1), 
fill=c("red"))
dev.off()
\end{lstlisting}


\begin{figure}[h]
	\center{\includegraphics[width=1\linewidth]{../img/hist_with_unif_dens.png}}
	\caption{Совмещённый график гистограммы и плотности равномерного закона.}
	\label{ris:hist_with_unif_dens}
\end{figure}

\newpage
\section{Эмпирическая функция распределения}
Эмпирическая CDF (Cumulative Distribution Function) является непараметрической оценкой неизвестной генеральной функции распределения случайной величины. При ее построении выборку $X$ предварительно сортируют по возрастанию наблюдаемых величин, каждому исходному элементу присваивают значения вероятности $1/n$, и далее на каждом шаге с использованием интерполяции вычисляют сумму этих вероятностей:
\begin{lstlisting}
> n <- length(df$V1)
> x <- sort(df$V1)
> vals <- unique(x)
> rval <- approxfun(vals, 
+         cumsum(tabulate(match(x, vals)))/n, 
+         method = "constant", yleft = 0, yright = 1, f = 0,
+         ties = "ordered")
\end{lstlisting}

Эмпирическая CDF определяется как
\begin{equation*}
	\hat{F}_{n}(x)=\hat{P}_{n(X \leq x)}=n^{-1} \sum_{i=1}^{n} I\left(x_{i} \leq x\right),
\end{equation*}
где $I$ -- индикаторная функция, принимающая два возможных значения:
\begin{equation*}
	I\left(x_{i} \leq x\right)=\left\{\begin{array}{ll}{1,} & {\textup { если } x_{i} \leq x} \\ {0,} & {\textup { если } x_{i}>x}\end{array}\right.
\end{equation*}

Согласно теореме Гливенко-Кантели, $\hat{F_n}(x)$ является состоятельной оценкой $F_n(x)$ и равномерно сходится к ней при $n \rightarrow \infty$.

На практике построение эмпирической CDF может оказаться полезным по следующим причинам:
\begin{itemize}
	\item если совокупный размер исходных выборок достаточно велик, то CDF является аппроксимацией "истинной" функции распределения, что может быть полезным для проверки статистических гипотез;
	\item график эмпирической CDF можно визуально сравнить с аналогичными кривыми для часто используемых теоретических распределений и проверить, не распределена ли случайная величина по известному закону;
	\item CDF может графически отобразить насколько "быстро" (с оценкой коэффициента угла наклона отрезков прямых на графике) вероятности увеличиваются от 0 до 1;
\end{itemize}

Построим совмещённые графики эмпирической и теоретической функций распределения:
\begin{lstlisting}
> png(filename = "../img/emp_and_theor_CDF.png", 
+     width = 1920, height = 1080,
+     pointsize = 24, res = 96 * 1.25)
> plot(ecdf(x), 
+      main = "Empirical and theoritical distribution functions", 
+      lwd = 3, col = "lightgreen")
> lines (x, (1:n)/n, type = 's', 
+        col="blue", lwd = 2)
> curve(punif(x, 2.5978, 17.9383), 2.5978, 17.9383, 
+       xlim = c(0,20), add = T, 
+       col = "red", lwd = 3)
> legend("bottomright", 
+        c("emperical CDF", "theoritical CDF"), 
+        lty = 1, col = c("blue", "red"))
> dev.off()
\end{lstlisting}

\begin{figure}[h]
	\center{\includegraphics[width=1\linewidth]{../img/emp_and_theor_CDF.png}}
	\caption{Совмещённый график эмпирической и теоретической функций распределения.}
	\label{ris:emp_and_theor_CDF}
\end{figure}

Из Рис.~\ref{ris:emp_and_theor_CDF} мы видим, что наши данные хорошо аппроксимируются полученной ранее предполагаемой функцией распределения.

\section{Выводы}
Основываясь на полученных в пунктах $1-4$ данных, целесообразно заключить, что исходные данные подчиняются равномерному распределению с параметрами $a = 2.597, b = 17.938$ и соответствующей плотностью распределения:
\begin{equation*}
	f_X(x) = 
		\begin{cases*}
		\frac{1}{15.341}, & x $\in$ [2.597; 17.938]  \\
		0, & x $\notin$ [2.597; 17.938]. \\
	\end{cases*}
\end{equation*}

В итоге проделанной работы на предоставленных в таблице исходных данных, были:
\begin{itemize}
	\item определены крайние члены вариационного ряда,
	\item определено оптимальное количество интервалов, на которые разбивается наблюдаемый диапазон изменения случайной величины при построении гистограммы плотности её распределения, по правилу Стёрджеса,
	\item определена ширина соответствующих интервалов разбиения для гистограммы,
	\item построена гистограмма частот нашей выборки,
	\item определены относительные частоты для каждого разбиения,
	\item определены меры центральной тенденции и меры изменчивости,
	\item произведён предварительный анализ наших данных с целью выбора оптимальной модели распределения с помощью построения графика Cullen and Frey skewness-kutorsis, а так же с помощью анализа критериев Колмогорова-Смирнова и Крамера-Мизеса,
	\item выбрана наиболее оптимальная модель, основываясь на предварительном анализе данных -- модель, соответствующая равномерному распределению,
	\item оценены параметры соответствующего распределения методом моментов,
	\item построены совмещённые графики гистограммы и плотности предполагаемого закона,
	\item определена эмпирическая функция распределения,
	\item построены совмещённые графики эмпирической и предполагаемых функций распределения,
	\item сделан окончательный вывод о распределении исходных данных.
\end{itemize}




\end{document}