\documentclass[14pt,a4paper]{scrartcl}
\usepackage{cmap}
\usepackage[utf8]{inputenc}
\usepackage[T1,T2A]{fontenc}
\usepackage[english,russian]{babel}
\usepackage{relsize}
\usepackage{graphicx}
\usepackage{subfigure}
\usepackage{mathtools}
\usepackage{amssymb}
\usepackage{float}
\usepackage{sidecap}
\usepackage{wrapfig}
\usepackage{caption}
\usepackage[table,xcdraw]{xcolor}
\usepackage{listings}
\usepackage{minted}
\usepackage{physics}
\usepackage[hidelinks]{hyperref}

\begin{document}
	\begin{titlepage}
	\begin{center}
		\large
		МИНИСТЕРСТВО ОБРАЗОВАНИЯ И НАУКИ\\ РОССИЙСКОЙ ФЕДЕРАЦИИ
		
		\vspace{0.5cm}
		
		МГТУ им Н.Э.Баумана
		\vspace{0.25cm}
		
		Факультет ФН
		
		Кафедра вычислительной математики и математической физики
		\vfill
		
		
		Соколов Арсений Андреевич\\
		\vfill
		
		
		{\LARGE Курсовая работа по дифференциальной геометрии\\[2mm]
		}
		\bigskip
		
		3 курс, группа ФН11-53Б\\
		Вариант 8
	\end{center}
	\vfill
	
	\newlength{\ML}
	\settowidth{\ML}{«\underline{\hspace{0.7cm}}» \underline{\hspace{2cm}}}
	\hfill\begin{minipage}{0.4\textwidth}
		Преподаватель\\
		\underline{\hspace{3cm}} Е.\,В.~Осипов\\
		«\underline{\hspace{0.7cm}}» \underline{\hspace{1.71cm}} 2019 г.
	\end{minipage}%
	\bigskip
	
	
	\vfill
	
	\begin{center}
		Москва, 2019 г.
	\end{center}
\end{titlepage}
\section{Теория}
В механике и особенно в релятивистской физике тензоры широко применяют в $n$-мерных римановых пространствах, являющихся более общими, чем евклидовы. Дадим определение этих пространств, а затем покажем, как конструируются тензоры в них. Начнём с основополагающего понятия римановых пространств - элементарного многообразия.

\subsection{Элементарное многообразие}
\textbf{Определение 1.} Элементарным $n$-мерным многообразием называют такое множество $M^n$, каждой точке которого
взаимнооднозначно поставлен в соответствие упорядоченный набор чисел $(X_1...X_n)$ из некоторой связной области $\mathcal{D} \in \mathbb{R}^n$, т.е, задано биективное отображение $\varphi: M^n \longrightarrow \mathcal{D} \in \mathbb{R}^n$.

Координатами точки $\mathcal{M} \in M^n$ в системе координат $\mathcal{D}$ называют координаты $X^i \in \mathbb{R}^n$ ее образа $\varphi(\mathcal{M})$, изменяющиеся в области $\mathcal{D} \in \mathbb{R}^n$.
Если для множества $M^n$ имеется другое биективное отображение $\varphi': M^n \longrightarrow \mathcal{D} \in \mathbb{R}^n$, то координаты точки $\mathcal{M}$ в системах координат $\mathcal{D}$ и $\mathcal{D'}$, связаны соотношениями:

\begin{equation}\label{e1}
	X'^i = X'^i(X^j), \quad i,j=1\dots n,
\end{equation}

которые предполагают число раз дифференцируемыми и невырожденными, т.е. $\det (\pdv{X'^i}{X^j}) \neq 0, \forall X^i \in \mathcal{D}$.
Введём обозначения для якобиевых матриц преобразования, а также для их производных:

\begin{equation}\label{eq2}
	Q_{\; j}^{i} \equiv\left(\frac{\partial X'^{i}}{\partial X^{j}}\right), \quad P_{\; j}^{i} \equiv\left(\frac{\partial X^{i}}{\partial X'^{j}}\right), \quad P_{\; j k}^{i} \equiv \frac{\partial^{2} X^{i}}{\partial X'^{j} \partial X'^{k}},
\end{equation}

и кроме того будем использовать обозначения для частных производных:

\begin{equation}\label{eq3}
	\pdv{f}{X^i} \equiv f_{,i}, \quad \pdv{f}{X'^i} \equiv f_{|i} = P_{\; i}^j f_{,i}.
\end{equation}

Примером двумерного $(n = 2)$ элементарного многообразия $M^2$ являются поверхности в $\mathbb{R}^3$, на которых определены криволинейные координаты $X_1, X_2$ и которые заданы тремя функциями:

\begin{equation}\label{eq4}
	x^i = x^{i}(X^1, X^2), \quad i=1,2,3.
\end{equation}

\subsection{Касательное пространство}
\textbf{Определение 2.} Кривой $\mathcal{L}$ в многообразии $M^n$ называют отображение $\mathcal{L}: [\xi_1,\xi_2] \in \mathbb{R}^1 \longrightarrow M^n$, которое записывают в виде функции:
\begin{equation}\label{eq5}
	X^i = X^i(\xi) \quad \forall \xi \in [\xi_1, \xi_2], \quad X^i \in M^n.	
\end{equation}

Здесь $X^i$ - координаты точки $\mathcal{M} \in M^n, [\xi_1, \xi_2]$ - некоторый отрезок из $\mathbb{R}^1, (\xi_1 < \xi_2)$, а функции (\ref{eq5}) предполагаем непрерывно дифференцируемыми, по крайней мере, два раза.


Зафиксировав значение параметра $\xi \in [\xi_1,\xi_2]$, получим некоторую точку $\mathcal{M} \in \mathcal{L}$, в ней можно вычислить производные от функций (\ref{eq5}):
\begin{equation}\label{eq6}
	a^i = \dv{X^i}{\xi}
\end{equation}

\textbf{Определение 3.} Упорядоченный набор $(a_1\dots a_n)$ производных (\ref{eq6}) называют компонентами касательного вектора $a^i$ в точке $\mathcal{M}$ кривой $\mathcal{L}$ в $ M^n $.

Если перейти к координатам $X'^i$ той же точки $\mathcal{M} \in \mathcal{L}$, то согласно (\ref{e1}) получаем, что компоненты касательного вектора $a'^i$ в этой системе координат будут иметь вид: $a'^i = \dv{X'^i}{\xi}$ и связаны с $a^i$ тензорным законом:
\begin{equation}\label{eq7}
	a'^i = Q_{\; j}^i a^j.
\end{equation}
Поскольку через фиксированную точку$\mathcal{M} \in M^n$ можно провести различные кривые $\mathcal{L}$, то, вообще говоря, в каждой точке $\mathcal{M}$ имеется множество упорядоченных наборов $(a_1\dots A_n)$. Определим операции с этими наборами.

Пусть имеется две кривые $\mathcal{L_\textup{1}}$ и $\mathcal{L_\textup{2}}$, заданные в виде функций $X_1^i(\xi), X_2^i(\xi)$, проходящие через точку $\mathcal{L}$, тогда можно построить два набора компонент касательных векторов $a_1^i = \dv{X_1^i}{\xi}$ и $a_2^i = \dv{X_2^i}{\xi}$.

Суммой компонент двух касательных векторов назовём набор
\begin{equation}\label{eq8}
	a_1^i + a_2^i = \dv{X_1^i + X_2^i}{\xi},
\end{equation}
который представляет собой компоненты касательного вектора к кривой $(X_1^i + X_2^i)(\xi)$ в данной точке $\mathcal{M}$.

Аналогично определяем произведение компонент $а^i$ на вещественное число $\lambda$:
\begin{equation}\label{eq9}
	\lambda a^i = \lambda \dv{X^i}{\xi} = \dv{\lambda X^i}{\xi}
\end{equation}

Поскольку набор чисел $(a_1 ...a_n)$ является элементом пространства $\mathbb{R}$, то, выбрав базис $e_i$ в этом пространстве, можно построить сам касательный вектор $a$ в точке $\mathcal{M}$ кривой $\mathcal{L}: a = a^ie_i = a'^ie'_i$, где $e'_i = P_{\; i}^j e_j$ - новый базис.


\textbf{Определение 4.} Касательным пространством в данной точке $\mathcal{M}$ элементарного многообразия $M^n$ называют множество касательных векторов $а = a^ie_i$, построенных ко всевозможным кривым $\mathcal{L}$, проходящим через данную точку.


\textbf{Теорема 1.} Касательное пространство в любой точке $\mathcal{M} \in M^n$ является $n$-мерным линейным пространством, которое обозначают как $T_{\mathcal{M}}M^n$, а векторы $e$, образуют базис в нем.


\subsection{Определение риманова пространства}
\textbf{Определение 5.} Элементарное $n$-мерное многообразие $M^n$ называют римановым пространством $\mathbb{V}^n$, если в каждой точке $\mathcal{M} \in M^n$ с координатами $X^i$ задана матрица $g_{ij}$ $n$-го порядка, которая является
\begin{enumerate}
	\item симметричной,
	\item невырожденной: $\det (\tilde{g}_{ij}) \neq 0, \quad \forall X^i$,
	\item компоненты ее являются непрерывно-дифференцируемыми функциями,
	\item при переходе к другим координатам $X'^l$ преобразуется по тензорному закону:
	\begin{equation}\label{eq10}
		g_{ij} = Q_{\; i}^k Q_{\; j}^l g'_{kl}.
	\end{equation}
\end{enumerate}


Двумерные поверхности в $\mathbb{R}$, очевидно, можно рассматривать как двумерные римановы пространства $\mathbb{V}^2$ с метрической матрицей $\tilde{g}_{IJ}$.











\end{document}